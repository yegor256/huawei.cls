% (The MIT License)
%
% Copyright (c) 2021 Yegor Bugayenko
%
% Permission is hereby granted, free of charge, to any person obtaining a copy
% of this software and associated documentation files (the 'Software'), to deal
% in the Software without restriction, including without limitation the rights
% to use, copy, modify, merge, publish, distribute, sublicense, and/or sell
% copies of the Software, and to permit persons to whom the Software is
% furnished to do so, subject to the following conditions:
%
% The above copyright notice and this permission notice shall be included in all
% copies or substantial portions of the Software.
%
% THE SOFTWARE IS PROVIDED 'AS IS', WITHOUT WARRANTY OF ANY KIND, EXPRESS OR
% IMPLIED, INCLUDING BUT NOT LIMITED TO THE WARRANTIES OF MERCHANTABILITY,
% FITNESS FOR A PARTICULAR PURPOSE AND NONINFRINGEMENT. IN NO EVENT SHALL THE
% AUTHORS OR COPYRIGHT HOLDERS BE LIABLE FOR ANY CLAIM, DAMAGES OR OTHER
% LIABILITY, WHETHER IN AN ACTION OF CONTRACT, TORT OR OTHERWISE, ARISING FROM,
% OUT OF OR IN CONNECTION WITH THE SOFTWARE OR THE USE OR OTHER DEALINGS IN THE
% SOFTWARE.

\documentclass[nobrand,nosecurity]{./huawei}
\renewcommand*\thetitle{\LaTeX{} Class \ff{huawei}}
\renewcommand*\thesubtitle{User's Guide}
\renewcommand*\theauthor{\nospell{Yegor Bugayenko}}
\begin{document}
\maketitle

\ff{Version: 0.0.0}
\newline
\ff{Date: 00.00.0000}

\section{Overview}

The provided class \ff{huawei} helps you design your work
documents and presentations keeping the code short and the
style elegant enough both for management and technical
papers. To use the class you simply mention its name
in the preamble:

\begin{minted}{text}
\documentclass{huawei}
\begin{document}
Hello, world!
\end{document}
\end{minted}

The document rendered from this \LaTeX{} code will look similar
to the document you are reading now. We recommend you to use
\ff{latexmk} to compile your \ff{.tex} files to \ff{.pdf}.
The simplest setup will require a few files staying next to your
\ff{.tex} file, in the same directory (``story'' is the name
of your project here):

\begin{minted}{text}
story\
  .latexmkrc
  .gitignore
  Makefile
  story.tex
\end{minted}

The content of your \ff{.latexmkrc} file would be this:

\begin{minted}{text}
$pdflatex = 'pdflatex %O --shell-escape %S';
\end{minted}

Make sure the \ff{.gitignore} file lists all the files generated
by \ff{pdflatex} during the compilation. You don't need to commit
them to your repository, since they are temporary and will
be generated again when you compile your document.

The recommended content of the \ff{Makefile} would be this:

\begin{minted}{text}
TEXS=$(wildcard *.tex)
PDFS=$(TEXS:.tex=.pdf)
all: $(PDFS)
%.pdf: %.tex
  latexmk -pdf $<
\end{minted}

In order to compile the document, just say \ff{make} on the command line.

\section{Layout Options}

There are a few class options, provided in square brackets after the
\ff{\textbackslash{}documentclass}, which can help you fine-tune
the layout of your document:

\ff{landscape}
  makes the document in landscape format, also changing the size
  of the paper to 16x9 inches (the default page size is
  \href{https://en.wikipedia.org/wiki/Paper_size}{A4}),
  making it perfect for presentations.

\ff{nocover}
  avoid printing the cover images on the first page by the
  \ff{\textbackslash{}PrintTitlePage} command.

\ff{anonymous}
  removes the name of the author everywhere, including the bottom
  of the page, where the author's name stays next to the name of the
  company.

\ff{nobrand}
  avoid mentioning the brand of Huawei anywhere
  in the document and removes the logo too.

\ff{nosecurity}
  avoids mentioning the level of security at the right top
  corner of the document and also avoids showing the ID of the author
  where it usually is visible.

\ff{nodate}
  don't show the date and time at the bottom of each page,
  where they usually are rendered in ISO~8601 format.

\ff{nopaging}
  avoids page numbers at the bottom of each page.

\section{Preamble}

In the preamble you can specify meta information about the document,
such as its title or author's name, here is how:

\begin{minted}{text}
\documentclass{huawei}
\renewcommand*\thetitle{Making Compression 15% Faster}
\renewcommand*\thesubtitle{Technical Report}
\renewcommand*\theauthor{Yegor Bugayenko}
\begin{document}
\maketitle
Hello, world!
\end{document}
\end{minted}

It's recommended to use \ff{\textbackslash{}renewcommand*} instead of
\ff{\textbackslash{}renewcommand} in order to let \LaTeX{} catch you
if by mistake a new line gets into the content.

The following meta commands are defined:

\ff{\textbackslash{}thetitle} is the main title of the document
  to be used in the text and in the properties of the PDF document.

\ff{\textbackslash{}thesubtitle} is the subtitle to be
  printed under the title.

\ff{\textbackslash{}theauthor} is the author of the document
  in ``first-name last-name'' format.

\ff{\textbackslash{}theid} is the internal ID of the author, if
  it's applicable.

\ff{\textbackslash{}thesecurity} is the level of security of
  the document, which is usually printed at the top right
  corner of it; usual values are ``Internal,''
  ''Confidential,'' or ``Secret.''

Default values of all these commands are empty. If you don't
renew them in your document, nothing will be printed.

\section{Custom Commands}

There is a number of supplementary commands for better text formatting,
which we introduced:

\ff{\textbackslash{}ff\{text\}}
  makes the text fixed-font with a nice border around.

\ff{\textbackslash{}tbd\{text\}}
  highlights the text, which is expected to be improved later
  (tbd stands for ``To Be Determined''), like \tbd{this one}.

Inside the document body you can use these commands:

\ff{\textbackslash{}PrintFirstPage\{front-image\}}
  prints the first page of a project charter or a similar landscape documents,
  placing the image \ff{front-image.pdf} on the front (the file should be present
  in the current dir. If you don't have the front image file, just leave
  the first argument empty.

\ff{\textbackslash{}PrintLastPage\{\}}
  prints the last page of a project charter or a similar landscape document.

\ff{\textbackslash{}PrintThankYouPage\{\}}
  prints the last page with a "Thank You" message in the center.

\ff{\textbackslash{}PrintDisclaimer\{\}}
  prints a paragraph at the bottom of the page with a standard disclaimer.

\section{Best Practices}

You are free to design your documents any way you want. However,
it would be convenient for yourself and for your readers, if you follow
the convention we have for business and technical documents.
The \ff{samples} directory contains a number of sample documents, which
we suggest you to use as templates when you start making new documents.

The rule of thumb is simple: try \emph{not} to format your documents.
Instead, let the class designed by us do this work for you. Just type
the content without changing the layout, adding colors, changing fonts,
etc. The less you modify the look-and-feel, the better your documents
will be perceived by your readers.

\subsection{Two Columns}

In the landscape format it's recommended to use two columns, for better
readability of the text. Here is how:

\begin{minted}{text}
\documentclass{huawei}
\begin{document}
\newpage
\begin{multicols}{2}
\section*{First}
Here goes the first column content.
\columnbreak
\section*{Second}
Here goes the second column content.
\end{multicols}
\end{document}
\end{minted}

A more complete example is in the \ff{samples/huawei-charter.tex}.

\subsection{Crumbs}

When you need to put many small information pieces into one page,
we recommend you to use ``crumbs'':

\begin{minted}{text}
\documentclass{huawei}
\begin{document}
\newpage
\section*{Project Details}
\begin{multicols}{2}
\raggedright
\crumb{Budget}{\$100K}

\crumb{Duration}{5 months}
\end{multicols}
\end{document}
\end{minted}

A more complete example is in the \ff{samples/huawei-charter.tex}.

\subsection{Code Blocks}

When you need to show some source code, we recommend to use
\ff{minted} environment, for example:

\begin{minted}{text}
\begin{minted}{text}
void foo() {
  return "Hello, world!";
}
\ end{minted}
\end{minted}

This is what you will see:

\begin{minted}{text}
void foo() {
  return "Hello, world!";
}
\end{minted}

\end{document}