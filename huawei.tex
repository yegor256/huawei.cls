\documentclass[nosecurity]{./huawei}
\title{Tutorial for the \texttt{huawei} \LaTeX Class}
\author{Yegor Bugayenko}
\begin{document}

\section{Overview}

Test here

\section{Class Options}

Class options, provided in square brackets after the
\ff{\textbackslash{}documentclass}, are the following:

\ff{landscape}
  makes the document in landscape format, also changing the size
  of the paper to 16x9 inches (the default page size is
  \href{https://en.wikipedia.org/wiki/Paper_size}{A4}),
  making it perfect for presentations.

\ff{anonymous}
  removes the name of the author

\ff{nobrand}
  avoids mentioning the brand of Huawei (removes the logo too)

\ff{nosecurity}
  avoids mentioning the level of security

\ff{nodate}
  don't show the date and time at the bottom of each page

\ff{nopaging}
  avoids page numbers at the bottom of each page

\section{Commands}

There is a number of supplementary commands for better text formatting,
which we recommend using:

\ff{\textbackslash{}ff\{text\}}
  makes the text fixed-font with a nice border around

\ff{\textbackslash{}href\{url\}\{text\}}
  makes a hyper link to the URL

Inside the document body you can use these commands:

\ff{\textbackslash{}PrintFirstPage\{front-image\}}
  prints the first page of a project charter or similar landscape documents,
  placing the image \ff{front-image.pdf} on the front (the file should be present
  in the current dir.

\ff{\textbackslash{}PrintLastPage\{\}}
  prints the last page of a project charter or similar landscape document.

\ff{\textbackslash{}PrintThankYouPage\{\}}
  prints the last page with a "Thank You" message in the center.

\ff{\textbackslash{}PrintDisclaimer\{\}}
  prints a paragraph at the bottom of the page with a disclaimer.

\end{document}