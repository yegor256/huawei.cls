\documentclass[nobrand,nosecurity]{./huawei}
\renewcommand*\thetitle{\LaTeX{} Class \texttt{huawei}}
\renewcommand*\thesubtitle{User's Guide}
\author{Yegor Bugayenko}
\begin{document}

\maketitle

\section{Overview}

The class \ff{huawei} provided helps you design your work
documents and presentations keeping the code short and the
style elegant enough both for management and technical
papers. To use the class you simply mention its name
in the preamble:

\begin{minted}{text}
\documentclass{huawei}
\begin{document}
Hello, world!
\end{document}
\end{minted}

The document rendered from this \LaTeX{} code will look similar
to the document you are reading now. We recommend you to use
\ff{latexmk} to compile your \ff{.tex} files to \ff{.pdf}.
The simplest setup will require a few files staying next to your
\ff{.tex} file, in the same directory (``story'' is the name
of your project here):

\begin{minted}{text}
story\
  .latexmkrc
  .gitignore
  Makefile
  story.tex
\end{minted}

The content of your \ff{.latexmkrc} file would be this:

\begin{minted}{text}
$pdflatex = 'pdflatex %O --shell-escape %S';
\end{minted}

This code should stay in \ff{.gitignore}:

\begin{minted}{text}
_minted-*
*.bbl
*.bcf
*.blg
*.fdb_latexmk
*.fls
*.log
*.run.xml
*.aux
*.out
\end{minted}

The recommended content of the \ff{Makefile} would be this:

\begin{minted}{text}
TEXS=$(wildcard *.tex)
PDFS=$(TEXS:.tex=.pdf)
all: $(PDFS)
clean:
  rm -rf *.aux *.bbl *.bcf *.blg *.fdb_latexmk *.fls *.log *.run.xml *.out
  rm -rf $(PDFS)
  rm -rf _minted*
%.pdf: %.tex
  latexmk -pdf $<
\end{minted}

\section{Layout Options}

There are a few class options, provided in square brackets after the
\ff{\textbackslash{}documentclass}, which can help you fine-tune
the layout of your document:

\ff{landscape}
  makes the document in landscape format, also changing the size
  of the paper to 16x9 inches (the default page size is
  \href{https://en.wikipedia.org/wiki/Paper_size}{A4}),
  making it perfect for presentations.

\ff{anonymous}
  removes the name of the author everywhere, including the bottom
  of the page, where the author's name stays next to the name of the
  company.

\ff{nobrand}
  avoid mentioning the brand of Huawei anywhere
  in the document and removes the logo too.

\ff{nosecurity}
  avoids mentioning the level of security at the right top
  corner of the document and also avoids showing the ID of the author
  where it usually is visible.

\ff{nodate}
  don't show the date and time at the bottom of each page,
  where they usually are rendered in ISO~8601 format.

\ff{nopaging}
  avoids page numbers at the bottom of each page.

\section{Preamble}

In the preamble you can specify meta information about the document,
such as its title or author's name, here is how:

\begin{minted}{text}
\documentclass{huawei}
\renewcommand*\thesubtitle{Making Compression 15% Faster}
\renewcommand*\thesubtitle{Technical Report}
\renewcommand*\theauthor{Yegor Bugayenko}
\renewcommand*\theid{00112233}
\renewcommand*\thesecurity{Internal Use Only}
\begin{document}
\maketitle
Hello, world!
\end{document}
\end{minted}

It's recommended to use \ff{\textbackslash{}renewcommand*} instead of
\ff{\textbackslash{}renewcommand} in order to let \LaTeX catch you
if by mistake a new line gets into the content.

\section{Custom Commands}

There is a number of supplementary commands for better text formatting,
which we introduced:

\ff{\textbackslash{}ff\{text\}}
  makes the text fixed-font with a nice border around

Inside the document body you can use these commands:

\ff{\textbackslash{}PrintFirstPage\{front-image\}}
  prints the first page of a project charter or similar landscape documents,
  placing the image \ff{front-image.pdf} on the front (the file should be present
  in the current dir.

\ff{\textbackslash{}PrintLastPage\{\}}
  prints the last page of a project charter or similar landscape document.

\ff{\textbackslash{}PrintThankYouPage\{\}}
  prints the last page with a "Thank You" message in the center.

\ff{\textbackslash{}PrintDisclaimer\{\}}
  prints a paragraph at the bottom of the page with a disclaimer.

\section{Best Practices}

\subsection{Two Columns}

In the landscape format it's recommended to use two columns, for better
readability of the text. Here is how:

\begin{minted}{text}
\documentclass{huawei}
\begin{document}
\newpage
\begin{multicols}{2}
\section*{First}
Here goes the first column content.
\columnbreak
\section*{Second}
Here goes the second column content.
\end{multicols}
\end{document}
\end{minted}

A more complete example is in the\ff{samples/charter.tex}.

\subsection{Crumbs}

When you need to put many small information pieces into one page,
we recommend you to use ``crumbs'':

\begin{minted}{text}
\documentclass{huawei}
\begin{document}
\newpage
\section*{Project Details}
\begin{multicols}{2}
\raggedright
\crumb{Budget}{\$100K}

\crumb{Duration}{5 months}
\end{multicols}
\end{document}
\end{minted}

A more complete example is in the\ff{samples/charter.tex}.

\end{document}